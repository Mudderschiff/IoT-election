\chapter{Background}
\section{Cryptography}

Cryptography is the science of securing information through encryption. Encryption or ciphering refers to the process of making a message incomprehensible \cite[18]{crypto} The security of all cryptographic methods is essentially based on the difficulty of guessing a secret key or obtaining it by other means. It is possible to guess a key, even if the probability becomes very small as the length of the key increases. It must be pointed out that there is no absolute security in cryptography \cite[25]{crypto}.

Practically all cryptographic methods have the task of ensuring one of the following security properties are met \cite[18]{crypto}. 
\begin{itemize}
    \item \textbf{Confidentiality} The aim of confidentiality is to make it impossible or difficult for unathorized persons to read a message \cite[18]{crypto}.
    \item \textbf{Authenticity} Proof of identity of the message sender to the recipient, i.e. the recipient can be sure that the message does not originate from another (unauthorized) sender \cite[18]{crypto}
    \item \textbf{Integrity} The message must not be altered (by unauthorized persons) during transmission. It retains its integrity \cite[18]{crypto}.
    \item \textbf{Non-repudiation} The sender cannot later deny having sent a message \cite[18]{crypto}.
\end{itemize}

Cryptographic algorithms are mathematical equations, i.e. mathematical functions for encryption and decryption \cite[19]{crypto}. A cryptographic algorithm for encryption can be used in a variety of ways in different applications. To ensure that an application always runs in the same and correct correct way, cryptographic protocols are defined. In contrast to the cryptographic algorithms, the protocols are procedures for controlling the flow of transactions for certain applications. \cite[22]{crypto}.

The idea of combining cryptographic methods with voting systems is not new. In 1981, David Chaum published a cryptographic technique based on public key cryptography that hides who a participant communicates with, aswell as the content of the communication. The untracable mail system requires messages to pass through a cascade of mixes (also known as a Mix Network) \cite[86]{chaum}. Chaum proposes that the techniques can be used in elections in which an individual can correspond with a record-keeping organisation or an interested party under a unique pseudonym. The unique pseudonym has to appear in a roster of accetable clients. A interested party or record keeping organisation can verify that the message was sent by a registered voter. The record-keeping organisation or the interested party can also verify that the message was not altered during transmission. \cite[84]{chaum}. 

In this use case, the properties of Confidentiality, Authenticity, Integrity and Non-repudiation are ensured. However, to be worthy of public trust, an election process must give voters and observers compelling evidence that the election was conducted properly without breaking voters confidentiality. The problem of public trust is further exacerbated to now having to trust election software and hardware, in addition to election officials, and proceedurs.

In 2021, the U.S. Election Assitance Commission (EAC) adopted the Voluntary Voting System Guidelines (VVSG) 2.0. \cite{eac-pressrelease}. The VVSG is intended for designers and manufacturers of voting systems. Currently, the VVSG is titled as "Recommendations to the EAC" because it's not yet the final version that voting system manufacturers will follow.
\cite{https://www.nist.gov/itl/voting/vvsg-introduction}. The VVSG 2.0 currently states only two methods for achieving software independence. The first through the use of independent voter-verifiable paper records, and the second through cryptographic E2E verifiable voting systems. The VVSG 2.0 states that a voting system need to be software independent through the use of independent voter-verifiable paper records
\cite[181]{https://www.eac.gov/sites/default/files/TestingCertification/Voluntary_Voting_System_Guidelines_Version_2_0.pdf}. However, due to the lack of E2E verifiable voting systems available within the current market, there are no verified E2E cryptographic protocols.\cite[199]{https://www.eac.gov/sites/default/files/TestingCertification/Voluntary_Voting_System_Guidelines_Version_2_0.pdf} The U.S. Election Assistance Commission, in collaboration with the National Institute of Standards and Technology initialised an Call for proposals to solicit, evaluate, and approve protocols used in E2E cryptographically verifiable voting systems. \cite{https://www.eac.gov/voting-equipment/end-end-e2e-protocol-evaluation-process}. 

Submitted protocols must support the following properties
\begin{itemize}
      \item Cast as Intended: Allow voters to confirm the voting system correctly interpreted their ballot selections while in the polling place via a receipt and provide evidence such that if there is an error or flaw in the interpretation of the voters’ selections.
      \item Recorded as Cast: Allow voters to verify that their cast ballots were accurately recorded by the voting system and included in the public records of encoded ballots.
      \item Tallied as Recorded: Provide a publicly verifiable tabulation process from the public records of encoded ballots.
\end{itemize}

\section{ElectionGuard}

One of the first pilots to see how E2E verifiable elections works in a real election took place in a district of Preston, Idaho, United States, on November 8, 2022. The Verity scanner from Hart InterCivic was used in this pilot, which was integrated with Microsoft's ElectionGuard. [EAC Report]. ElectionGuard is a toolkit that encapsulates cryptographic functionality and provides simple interfaces that can be used without cryptographic expertise. The principal innovation of ElectionGuard is the seperation of the cryptographic tools from the core mechanics and user interfaces of voting systems. In it's preferred deployment, ElectionGuard doesn't replace the existing vote counting infrastructure but instead runs alongside and produces its own independently-verifiable tallies \cite[1-2]{eg-paper}. The cryptographic design is largely inspired by the cryptographic voting protocol by Cohen (now Benaloh) and Fischer in 1985 and the voting protocol by Cramer, Gennaro and Schoenmakers in 1997 \cite[5]{eg-paper}. 

----------------------------------------------
The philosophy of ElectionGuard has been to cover the majority of voting scenarios with an approach that is as simple as possible to understand and verify. It can be used with Precinct Ballot Scanners, Electronic Ballot Markers, Internet Voting, Risk-Limiting Audits and even vote by mail and many more.

In all applications, an election using ElectionGuard begins with a key-generation ceremony in
which an election administrator works with guardians to form election keys. Later, usually at the
conclusion, the administrator will again work with guardians to produce verifiable tallies. What
happens in between, however, can vary widely. [ElectionGuard: a Cryptographic Toolkit
to Enable Verifiable Elections].

This thesis focuses on the implementation of the Key Generation Ceremony and the Guardians using the ESP32 microcontroller. The ESP32 is a low-cost, low-power system on a chip microcontroller. It is widely used in IoT applications. [source]
----------------------------------------------
\subsection{Election Verification}
ElectionGuard support the central aspects which a VVSG 2.0 compliant voting system must support Cast as Intended, Recorded as Cast and Tallied as Recorded. \cite[17]{eg-paper}. Key element supporting cast-as-intended and recorded-as-cast verifiability is through confirmation codes. Ballots can be ce challenged or cast and both are included in the election record. Voters can check if the expected confirmation code appears in the election records and for the challenged ballots, that it shows the correct selections \cite[18]{eg-paper}. Tallied-as-cast verifiability is supported through the inclusion of all ballots and decryption proofs in the election record. Any voter can verify that the ballot is accurately incorporated in the tally, and the decryption proofs demonstrate the validity of the announced tally \cite[18]{eg-paper}. To confirm the election's integrity, independent verification software can be used at any time after the completion of an election. \cite[6]{eg-paper}. Some, may choose even to write their own verifiers.  