\chapter*{Glossary}
\addcontentsline{toc}{chapter}{Glossary}

\begin{glossary}
    \newglossaryentry{eac}{
        name={EAC},
        description={Election Assistance Commission}
    }
    \newglossaryentry{e2e}{
        name={E2E},
        description={End-to-end}
    }
    \newglossaryentry{vvsg}{
        name={VVSG},
        description={Voluntary Voting System Guidelines}
    }
    \newglossaryentry{ballot box}{
        name={ballot box},
        description={A term used to represent collection of submitted ballots in programming code mimicking the physical item.}
    }
    \newglossaryentry{ballot code}{
        name={ballot code},
        description={A unique hash value generated by an encryption device to anonymously identify a ballot and allow the voter to confirm the ballot has been submitted. code is not to be confused with programming code, but is rather a code to confirm the ballot.}
    }
    \newglossaryentry{ballot decryption}{
        name={ballot decryption},
        description={Decrypting a secure encrypted ciphertext ballot to a readable plaintext ballot.}
    }
    \newglossaryentry{ballot encryption}{
        name={ballot code},
        description={Encrypting a readable plaintext ballot to a secure encrypted ciphertext ballot.}
    }
    \newglossaryentry{cast ballot}{
        name={cast ballot},
        description={cast ballot}
    }
    \newglossaryentry{ciphertext ballot}{
        name={ciphertext ballot},
        description={An encrypted representation of a voter's filled-in ballot.}
    }
    \newglossaryentry{ciphertext tally}{
        name={ciphertext tally},
        description={A ciphertext tally is the homomorphically-combined and encrypted representation of all selections made for each option on every contest in the election. It is an aggregate of all the encrypted cast ballots in the election. The ciphertext tally is the tally while it is is still in the encrypted state.}
    }
    \newglossaryentry{contest}{
        name={ciphertext tally},
        description={A contest in an manifest consists of a set of candidates or options together with a selection limit. Contests can be customized to suit the voting experience with options like approval voting, ranked-choice voting and write-ins. Contests can have special rules around selection limits to handle undervotes, overvotes, and null votes.}
    }
    \newglossaryentry{decryption share}{
        name={decryption share},
        description={A guardian's partial share of a ballot decryption or tally decryption.}
    }
    \newglossaryentry{election}{
        name={election},
        description={An election in ElectionGuard is an election as described by a manifest. An election will have one encryption key to encrypt ballots and after tally, will result in an election record.}
    }
    \newglossaryentry{guardian keys}{
        name={ guardian keys},
        description={A key pair owned by a specific guardian for joint encryption in combination with the other guardians to secure the election.}
    }
    \newglossaryentry{backup}{
        name={backup},
        description={A point on a secret election polynomial and commitments to verify this point for a designated guardian.}
    }
    \newglossaryentry{election polynomial}{
        name={election polynomial},
        description={The election polynomial is the mathematical expression that each guardian uses for encryption. Each guardian has a polynomial where the first coefficient is used to generate their election key pair. A different point associated with the polynomial is shared with each of the other guardians in the key ceremony so that the guardians can come together and decrypt in the tally ceremony}
    }
    \newglossaryentry{election record}{
        name={election record},
        description={A verifiable record of the public artifacts or files of the election. This includes items like the manifest and the encrypted ballots so an individual or third party can verify the election end-to-end.}
    }
    \newglossaryentry{encryption device}{
        name={encryption device},
        description={A device loaded with the election context that performs ballot encryption.}
    }
    \newglossaryentry{joint key}{
        name={joint key},
        description={A public encryption key which is the combination the public key of the election key pair of each of the guardians. This key is created as the last step in the key ceremony.}
    }
    \newglossaryentry{key ceremony}{
        name={key ceremony},
        description={The process conducted at the beginning of the election to create the joint key for ballot encryption during the election. In ElectionGuard, each guardian creates an election key pair and shares a recovery method for their decryption with the other election guardians as part of the key ceremony in case that particular guardian cannot attend the tally ceremony.}
    }
    \newglossaryentry{key pair}{
        name={key pair},
        description={A key pair consists of a linked private key and public key. Key pairs are used in public key cryptography, in which public keys are distributed to others to encrypt messages that only the private key can decrypt.}
    }
    \newglossaryentry{manifest}{
        name={key pair},
        description={The manifest is the information that uniquely specifies and describes the structure and type of the election, including geopolitical units, contests, candidates, ballot styles, etc. In ElectionGuard, it is a file that is created before running an election. The internal manifest is a wrapper around the manifest used in programming code to simplify and avoid processing the same information twice. Unlike the manifest, the internal manifest is not meant for serialization.}
    }
    \newglossaryentry{mediator}{
        name={mediator},
        description={A mediator is used to mediate communication (if needed) of information such as keys between the guardians. This can be a person but in ElectionGuard this often refers to the server mediating / coordinating between the guardian machines that maintains all the public information between the key and tally ceremonies.}
    }
    \newglossaryentry{plaintext ballot}{
        name={plaintext ballot},
        description={A plaintext tally is the summation of votes for each candidate for each contest in the election. The plaintext tally, or just tally, is the decrypted ciphertext tally or decrypted aggregate ballot which contains the election results.}
    }
    \newglossaryentry{post-election audit}{
        name={plaintext ballot},
        description={A post-election audit verifies that the voting equipment used to count ballots during an election properly counts a sample of voted ballots after an election}
    }
    \newglossaryentry{quorum}{
        name={plaintext ballot},
        description={The minimum count of guardians that must be present in order to successfully decrypt the tally.}
    }
    \newglossaryentry{selection}{
        name={selection},
        description={A selection or vote is the selected candidate(s) or option(s) in a contest on a voter's ballot.}
    }
    \newglossaryentry{spoiled ballot}{
        name={spoiled ballot},
        description={Spoiling allows a voter to turn in their ballot without their ballot being included in the election tally. A spoiled ballot is a ballot the voter has submitted as spoiled. The voter must submit a replacement ballot as their official cast ballot.In ElectionGuard, ballot spoiling is used as a means for voters to challenge the voting machine and force it to reveal the contents of spoiled ballots for public scrutiny. In this way, a spoiled ballot challenges the system and the term challenge ballot is used. The ballot is spoiled after encryption where the machine cannot modify the ballot. Since the ballot will not be used in the official tally, the content of the ballot can be revealed at the same time as the tally allowing a voter to verify the encryption and decryption process.}
    }
    \newglossaryentry{spoiled ballot}{
        name={spoiled ballot},
        description={The process of decrypting the encrypted tally to the decrypted tally. The guardians from the key ceremony who are available attend this ceremony. There must be at least enough guardians to meet the quorum. Each guardian will decrypt their decryption shares and their share for each missing guardian. These shares will then be combined to create the decrypted spoiled ballots and decrypted tally.}
    }
    \newglossaryentry{tally decryption}{
        name={tally decryption},
        description={Decrypting an encrypted ciphertext tally to a readable plaintext tally to view results. This is essentially the same concept as ballot decryption.}
    }
\end{glossary}