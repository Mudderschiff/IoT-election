\chapter*{Abstract}
\addcontentsline{toc}{chapter}{Abstract}
The rapid expansion of the \ac{IoT} introduces critical challenges in maintaining data confidentiality, integrity, and verifiability in resource-constrained environments.This thesis investigates the implementation of \ac{E2E} verifiable voting systems on IoT devices, focusing on the ElectionGuard 1.0 specification deployed on ESP32 microcontrollers. This work underscores the viability of ElectionGuard in constrained environments while highlighting critical areas for performance refinement. Key research questions explore computational efficiency (e.g., hardware accelerators for cryptographic operations, memory constraints, and processing bottlenecks) and communication efficiency (protocol selection, bandwidth optimization, and network latency mitigation).

A custom C port of ElectionGuard was developed to leverage ESP32's hardware features, including \ac{RSA} and \ac{SHA} accelerators, achieving a 4x speed improvement in cryptographic operations compared to software-only implementations. The system leverages \ac{MQTT} for communication, balancing payload size with reliable transmission via QoS levels. Protocol Buffers were used for efficient binary serialization, ensuring interoperability between the Python-based Tallier and C-based Guardian nodes. Network traffic analysis reveals consistent \ac{RTT} under load, although initial bursts suggest the need for congestion management in larger deployments. Homomorphic aggregation on the Tallier (implemented in Python) emerged as the primary bottleneck, limiting scalability due to its CPU-bound nature, requiring 251.9 seconds for 1000 ballots. 

