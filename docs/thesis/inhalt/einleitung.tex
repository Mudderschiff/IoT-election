\chapter{Introduction}

\ac{IoT} refers to a network of interconnected devices, objects, and systems embedded with sensors, software, and other technologies to collect and exchange data. These devices, ranging from smart appliances to industrial machines, enable communication and automation across various sectors \cite[1]{combinatorial}. However, the sharing of sensitive data in \ac{IoT} systems raises critical concerns about confidentiality and integrity, particularly due to limited computational resources, diverse standards, and network vulnerabilities \cite[1]{smpc}.

Data aggregation, the process of gathering and summarising information from multiple sources, is crucial for \ac{IoT} data analysis but introduces privacy risks. For instance, in smart metering systems, \ac{PPDA} is a leading solution for securing consumer data by securely aggregating meter readings at the gateways, preventing attackers from identifying individual user profiles \cite[2]{ppda-fog}. While various security techniques have been developed, \ac{PPDA} is considered more convenient. Many data aggregation schemes use cryptographic techniques, such as homomorphic encryption, to encrypt energy consumption data. Nevertheless, the computational intensity of these techniques often renders them impractical for resource-constrained \ac{IoT} devices \cite[113-114]{smart-meter}. \ac{PPDA} techniques also apply to other domains, such as electronic voting. Verifiable voting systems, for example, use homomorphic encryption to tally ballots while preserving voter anonymity. This method breaks the link between individual voters and their votes, keeping them secret \cite[53]{stuve-study}. However, centralised decryption by a single tallier risk exposing individual votes. The tallier who owns the decryption key can decrypt all individual votes and learn how each voter has voted. To mitigate this, threshold cryptography---a subfield of \ac{MPC}---distributes decryption keys among multiple talliers, requiring collaboration among the parties in order to decrypt the results \cite[40]{stuve-study}. While threshold schemes enhance security, their reliance on synchronised interactions introduce communication bottlenecks \cite[45]{stuve-study}. Verifiability and accountability are equally critical in \ac{PPDA} systems. Verifiability ensures the correctness of aggregated results (e.g., election outcomes) \cite[4]{stuve-study}, while accountability-a stronger notion of verifiability-enables precise identification of error sources \cite[10, 27]{stuve-study}. \ac{ZK} proofs, a foundational cryptographic tool, allow parties to validate computational steps without revealing sensitive data \cite[13]{stuve-study}, though they incur additional overhead. 

\section{Research Questions}
This thesis evaluates the implementation challenges and performance characteristics of \ac{E2E} verifiable voting systems in resource-constrained \ac{IoT} environments, focusing on a subset of the ElectionGuard 1.0 specification deployed on ESP32 microcontrollers. The following research questions guide this evaluation:

\textbf{Computational Efficiency:} 
How does the choice of hardware components (e.g., Tallier Node, Guardian Nodes) impact the performance of an IoT-based voting system running ElectionGuard? 
How feasible is it to run the existing ElectionGuard implementations on ESP32 microcontrollers, and what are the potential limitations in terms of memory usage and processing power? 
How do hardware accelerators on the ESP32 improve the performance of cryptographic operations in ElectionGuard? 
How does the computational load on the Tallier Node impact the overall performance of the voting system? 

\textbf{Communication Efficiency:} 
What communication protocol is suitable for reliably handling large data transfers on the ESP32 microcontroller? What strategies can be employed to reduce the bandwidth requirements?
What are the challenges in implementing communication protocols on resource-constrained \ac{IoT} devices like ESP32, and how can these be addressed? 
How does the network latency between the Tallier Node and Guardian Nodes impact the performance of the voting system?

\section{Structure of the Thesis}
The thesis is structured as follows. Chapter \ref{chap:background} establishes foundational knowledge required for understanding the cryptographic voting system, covering core cryptographic principles, verifiability concepts, and the technical components of the ElectionGuard 1.0 specification and ESP32 hardware. Chapter \ref{chap:implementation} details the implementation of an \ac{IoT} voting system's architecture, focusing on the integration of hardware, cryptographic processes, and communication protocols to support the election process. Chapter \ref{ch:conclusions} presents the conclusions of the thesis, summarising the findings related to the performance of ElectionGuard on ESP32 microcontrollers, and suggesting future work directions.